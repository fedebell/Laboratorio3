\documentclass[10pt,a4paper]{article}
\usepackage[utf8]{inputenc}
\usepackage[italian]{babel}
\usepackage{amsmath}
\usepackage{amsfonts}
\usepackage{amssymb}
\usepackage{graphicx}
\usepackage{gensymb}
\usepackage[left=2cm,right=2cm,top=2cm,bottom=2cm]{geometry}
\newcommand{\rem}[1]{[\emph{#1}]}

\author{Gruppo BN \\ Federico Belliardo, Marco Costa, Lisa Bedini}
\title{Ottica 1}
\begin{document}

\maketitle
\section{Scopo dell'esperienza}
Questa esperienza si divide in due parti: A e B.
La parte A è dedicata al calcolo della lunghezza d'onda del sodio, mentre nella parte B attraverso l'osservazione della luce emessa da tre lampade diverse si è misurato il passo reticolare dello spettroscopio a reticolo e la costante di Rydberg.

\section{Materiale occorrente}
\begin{itemize}
\item lampada al cadmio;
\item lampada al sodio;
\item lampada al mercurio;
\item lampada a idrogeno;
\item elemento dispersivo A: prisma;
\item elemento dispersivo B: reticolo;
\item supporto con goniometro integrato;
\end{itemize}
\\
Inoltre avremo a disposizione due telescopi, uno di raccolta della luce dotato di fenditura per regolare l'intensità del fascio, e uno di osservazione solidale a un supporto mobile dotato di goniometro, per misurare la posizione del telescopio rispetto all'elemento dispersivo.

\section{Parte A - Descrizione esperimento}
\subparagraph{Lampada al cadmio}
Per tarare l'apparato strumentale si è posta la lampada al cadmio in modo da allineare le fenditure, della lampada stessa e del telescopio di raccolta. Abbiamo regolato l'apertura del diaframma al minimo così da avere l'incertezza minima, quindi abbiamo posizionando l'elemento dispersivo con un angolo di almeno $60\degree$ rispetto alla direzione del telescopio di raccolta e osservato le righe di emissione del cadmio, di colori rosso, verde, azzurro e viola. Come ultima cosa abbiamo ruotato lentamente il prisma per trovare l'angolo di minima dispersione per la riga verde \footnote{Perché ha la lunghezza d'onda più vicina a quella del sodio.}, cioè l'angolo di inversione del moto delle righe dello spettro che si osserva muovendo il prisma. Non abbiamo eseguito una misura dell'angolo $alpha_0$ al quale viene focalizzato il fascio (come sarà invece eseguito nella prossima parte) poiché risulta essere una costante additiva inessenziale nella nostra analisi dati.
Si è dunque eseguita la calibrazione spettrale, nota le lunghezze d'onda delle principali righe di emissione del cadmio. Abbiamo misurato l'angolo di osservazione di ciascuna riga \footnote{In tutta la relazione ogni misura angolare è intesa come media delle tre misure eseguite dai tre componenti del gruppo ed è intesa avere come errore quello di lettura sulla scala graduata del nonio.} (vedi tabella \ref{cadmio}) e eseguito un fit lineare $y=ax+b$ nel grafico $\alpha\, \textit{vs}\, 1/\lambda$ ottenendo come parametri $a = -3250 \pm 40 \frac{\degree}{\,nm}$ e $ b = 148.1 \pm 0.1 \degree$ e $\chi^2=23/2$. Il $\chi^2$ ottenuto è alto a causa del fatto che la curva di calibrazione lineare in funzione di $\frac{1}{\lambda}$ è solo un'approssimazione, in realtà si ha comportamento molto più complesso per il prisma. I dati raccolti si trovano in tabella \ref{cadmio} mentre in figura \ref{pin} è presente il grafico con il \emph{fit}.

\begin{table}[!htb]
\centering
\begin{tabular}{|c|c|c|}
\hline 
Colore & $\lambda (\,nm)$ & $\alpha (\degree)$ \\
\hline
Rosso & $643.8\pm0.1$ & $143 \degree 1' \pm 1'$ \\ 
\hline 
Verde & $508.643 \pm 0.001$ & $141 \degree 45' \pm 1'$ \\ 
\hline 
Azzurro & $480.0 \pm 0.1$ & $141 \degree 19' \pm 1'$ \\ 
\hline 
Viola & $467.8 \pm 0.1$	& $141 \degree 5' \pm 1'$ \\ 
\hline 
\end{tabular} 
\caption{Righe di emissione del cadmio con relativa posizione angolare.}\label{cadmio}
\end{table}

\begin{figure}[!htb]
  \centering
  \includegraphics[scale=0.6]{imm.png}
\caption{Grafico $\alpha\, \textit{vs}\, 1/\lambda$ e \emph{fit}.}
\label{pin}
\end{figure}

Per eseguire i fit di tutta la relazione si è la funzione \emph{curvefit} della libreria \emph{pylab} con l'opzione \emph{$absolute\,sigma = "true"$}.\\

\subparagraph{Lampada al sodio}
Si è sostituita la lampada al cadmio con la lampada al sodio precedentemente accesa in modo che si stabilizzasse. Abbiamo individuato la riga\footnote{Un doppietto in realtà ma l'uso del prisma non consente tale risoluzione.} di emissione del sodio e misurata la sua posizione angolare, che risulta $\alpha_{Na}=142 \degree 35'\pm 1'$. Usando le relazioni ricavate precedentemente dal \emph{fit}, si calcola $\lambda_{Na}=590 \pm 10 \,nm$. Questo valore sarà confrontato con quello ottenuto usando il reticolo, elemento dispersivo con più potere risolutivo del prisma.

\section{Parte B - Descrizione esperimento}
\subparagraph{Lampada al mercurio}
Come prima operazione si esegue la procedura perla taratura dell'apparato sperimentale. Si è rimosso il reticolo dal suo supporto e successivamente abbiamo posto la lampada al mercurio e i telescopi in modo da allineare il sistema aiutandoci con al croce nera disegnata sulla lente del telescopio che ne indica il centro: da fare coincidere con il filamento della lampada osservato. Successivamente abbiamo regolato l'ampiezza della fenditura e il fuoco dei telescopi. In queste condizioni si misura un angolo $\alpha_0 = 169 \degree 45 '\pm 1'$ sulla ghiera. D'ora in avanti tutte le misure di angoli saranno riportate come $\alpha=\alpha_{lettura}-\alpha_0$. Abbiamo posto il reticolo con un angolo di almeno $60\degree$ rispetto al telescopio di raccolta, quindi abbiamo controllato che le righe di interesse, quelle della serie di Balmer, fossero ben visibili.\\
Come ultima operazione si è calcolato il passo reticolare $d$. Per fare ciò abbiamo misurato l'angolo di ordine zero della riga verde e l'angolo della stessa riga al primo ordine, i due risultati sono: $\alpha_V^{0} = 58 \degree 8 ' \pm 2'$ e $\alpha_V^{1} = 106 \degree 7' \pm 2'$. Da semplici considerazioni geometriche si ricavano le seguenti relazioni:
\begin{equation}
\theta_i=\frac{1}{2}(\pi-\alpha_V^{0})=61\degree 56'\pm2'
\end{equation}
\begin{equation}
\theta_d=\pi-\theta_i-\alpha_1= 12 \degree 56' \pm 3'
\end{equation}
Quindi è possibile calcolare $d$ sfruttando la relazione del reticolo 
\begin{equation}
d(sin{\theta_i}-sin{\theta_d})=m\lambda
\end{equation}
Si ottiene $d = 840.2 \pm 0.5 \,nm$ considerando nota la lunghezza d'onda del mercurio $\lambda_{Hg}=546.074\pm0.001\,\,nm$. Dal valore di $d$ si calcola il numero di righe per millimetro $N = 1190 \pm 3$, vicino ma non perfettamente in accordo con il valore nominale di 1200 (differisce da esso per $3 \Delta N$).

\subparagraph{Lampada a idrogeno}
Si è sostituita la lampada a mercurio con quella a idrogeno, senza modificare l'apertura della fenditura del telescopio. Abbiamo quindi misurato la distanza angolare delle righe di emissione per calcolarne le lunghezze d'onda mediante la formula $d(\sin{\theta_i}-\sin{\theta_d})=m\lambda$. I risultati sono riportati in tabella \ref{idrogeno}. Le righe verde e arancione non fanno parte dello spettro dell'idrogeno ma sono dovute ad impurità nella lampada pertanto i dati relativi a queste non saranno necessari nell'analisi dati, ma sono riportati nella tabella per completezza.\\


%PER MARCO: Gli errori degli angoli della tabella in teoria dovrebbero essere 2 perchè stro sottraendo l'angolo di riferimento.
\begin{table}[!htb]
\centering
\begin{tabular}{|c|c|c|}
\hline
Colore & $\alpha$ ($\degree$) & $\lambda (\,nm)$\\
\hline 
Viola & $98 \degree 2' \pm 1'$ & $432 \pm 2 \,nm$ \\ 
\hline 
Azzurro & $101 \degree 53' \pm 1'$ & $486 \pm 2 \,nm$ \\ 
\hline 
Verde (deb.)& $105 \degree 5' \pm 1'$ & $531 \pm 2 \,nm$ \\ 
\hline
Arancio & $110 \degree 48' \pm 1'$ & $613 \pm 2 \,nm$ \\ 
\hline 
Rossa & $113 \degree 30' \pm 1'$ & $653 \pm 1 \,nm$ \\ 
\hline 
%Viola (2° Ord.) & $113 \degree 35' \pm 1'$ & _ \\ 
%\hline 
%AGGIUNGERE ARANCIONE
\end{tabular} 
\caption{Posizione angolare delle righe dell'idrogeno e lunghezza d'onda.}\label{idrogeno}
\end{table}

Lo spettro dell'idrogeno è descritto dall'equazione di Rydberg:
\begin{equation}
\frac{1}{\lambda}= R \left( \frac{1}{n_{1}^2}-\frac{1}{n_{2}^2} \right)
\end{equation}

Per la serie di \emph{Balmer}, la più visibile, $n_1=2$; a questa serie appartengono la riga viola con $n_2=5$, quella azzurra con $n_2=4$ e quella rossa con $n_2=3$. Si è effettuato un fit lineare a un parametro ottenendo un valore per la costante di Rydberg pari a $R = 10.9 \pm 0.1 \frac{1}{\mu m}$ in accordo con il valore atteso pari a $R=10.968\, \frac{1}{\mu m}$, associato a questo fit abbiamo $\chi^2 = 0.44/1$.\\

\begin{figure}[!htb]
  \centering
  \includegraphics[scale=0.6]{ryd.png}
\caption{\emph{fit} per determinare la costante di Rydberg.}
\label{pin}
\end{figure}


\subparagraph{Lampada al sodio}
Si è sostituita la lampada a idrogeno con quella al sodio per misurare la lunghezza d'onda del doppietto giallo e confrontare tale dato con quello attenuto nella prima parte dell'esperienza. Al primo ordine di diffrazione abbiamo misurato i seguenti valori angolari $\alpha_{Na,1}= 108\degree 42'\pm 1' $ e $\alpha_{Na,2}=108\degree 45' \pm 1'$, quindi, sfruttando l'equazione del reticolo si ottiene la lunghezza d'onda $\lambda_{Na,1}=583.0 \pm 2 \,nm$ e $\lambda_{Na,2}=583.8 \pm 2 \,nm$.
Abbiamo abusato delle regole sulle cifre significative perché in tutta l'analisi dati gli errori sono sempre stati sovrastimati, ma si voleva mostrare la separazione tra le righe spettrali stimata di $0.8 \, \,nm$, contro quella attesa di $0.6 \, \,nm$, i due valori sono molto vicini.\\

\section{Conclusioni}
Abbiamo ottenuto un'ottima stima della costante di Rydberg, con un errore inferiore all'$1\%$. Anche la stima della lunghezza d'onda della luce gialla del sodio è sostanzialmente corretta rispetto al valore atteso di $590 \, \,nm$ circa. Tuttavia l'errore è evidentemente sovrastimato e rende la misura  compatibile con quella eseguita successivamente usando il prisma che è circa $583\,nm$ con un errore che la rende incompatibile con il valore noto. Si suppone che ciò sia dovuto ad un errore sistematico su $d$ causata dall'errata determinazione degli angoli a cui si avevano l'ordine 0 e 1 della riga verde del mercurio.\\

\end{document}