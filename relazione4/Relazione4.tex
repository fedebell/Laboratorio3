\documentclass[10pt,a4paper]{article}
\usepackage[utf8]{inputenc}
\usepackage[italian]{babel}
\usepackage{amsmath}
\usepackage{amsfonts}
\usepackage{amssymb}
\usepackage{graphicx}
\usepackage[left=2cm,right=2cm,top=2cm,bottom=2cm]{geometry}
\newcommand{\rem}[1]{[\emph{#1}]}

\author{Gruppo AC \\ Federico Belliardo, Giulia Franchi, Francesco Mazzoncini}
\title{Esercitazione N.4: Amplificatore a transistor}
\begin{document}

\maketitle

\section{Scopo dell'esperienza}
L'esercitazione ha come scopo quello di realizzare un circuito amplificatore, utilizzando un transistor \textit{npn} 2N1711.

\section{Montaggio del circuito e verifica del punto di lavoro}
Abbiamo montato il circuito in Fig.1 come richiesto, con: $R_1= \pm k\Omega$, $R_2= \pm k\Omega$, $R_C= \pm k\Omega$, $R_E= \pm k\Omega$, $C_{IN}= \pm nF$, $C_{OUT}= \pm nF$ e $C_E= \pm mF$. Tutti i componenti sono stati misurati con multimetro digitale.
Supponiamo che il transistor lavori in zona attiva, $V_{BE}= \pm V$. Abbiamo inoltre definito $V_{PART}=V_{CC}\frac{R_2}{R_1+R_2}$, per determinarla abbiamo prima misurato la tensione in ingresso $V_{CC}= \pm V$, così la tensione ai capi del partitore risulta $V_{PART}= \pm V$.


\subsection{Misura del punto di lavoro}
Per la misura del punto di lavoro del circuito si è misurato $V_{CE}= \pm V$ e la caduta di potenziale ai capi della resistenza $R_C$, $V_{RC}= \pm V$, in modo da poter determinare $I_C=\frac{V_{RC}}{I_C}= \pm mA$. Confrontando questi valori misurati con i valori teorici, $I_C=\frac{V_{PART}-V_{BE}}{R_E}= \pm mA$ e $V_{CE}=V_{CC}-(R_C+R_E)I_C= \pm V$, si vede che (sono compatibili o no?!).
La retta di lavoro attesa è $V_{CC}=V_{CE}+I_C(R_C+R_E)$, \rem i valori ottenuti sono compatibili?!.

\subsection{Misura delle tensioni ai terminali del transistor}
Abbiamo misurato le tensioni $V_B= \pm V$, $V_E= \pm V$, $V_{BE}= \pm V$ e $V_C= \pm V$. Sono compatibili con quanto abbiamo calcolato teoricamente?! $V_B= V_{PART}$, $V_E=I_C R_E$, $V_C=V_CC-I_C R_C$.

\subsection{Valutazione della corrente di base}
Ci aspetteremmo una corrente di base $I_B=\frac{I_C}{h_{FE}}= muA$, dato che si suppone il transistor lavori in zona attiva. Misuriamo le cadute di potenziale ai capi delle resistenze $R_1$ e $R_2$, $V_{R1}= \pm V$ e $V_{R2}= \pm V$, da cui abbiamo ricavato $I_{R1}= \pm muA$ e $I_{R2}= \pm muA$, dalle quali infine abbiamo ricavato $I_B=I_{R1}-I_{R2}= \pm muA$. Quindi il partitore è stiff?!

\section{Risposta a segnali sinusoidali di frequenza fissa}
In questa parte dell'esperienza si è utilizzato un segnale ad una frequenza fissa pari a $f= \pm Hz$.

\subsection{Misura del guadagno in tensione}
Abbiamo preso diverse misure di $V_OUT$ (prendiamo anche il suo sfasamento rispetto a Vin?) in funzione di $V_IN$ al variare di quest'ultima, prestando attenzione ai fenomeni di clipping. Nella tabella seguenze riportiamo le nostre misure, aggiungendo anche il calcolo del guadagno in tensione, $A_V=\frac{V_{OUT}}{V_{IN}}$.

\begin{table}[h]
\centering
\begin{tabular}{|c|c|c|c|c|c|}
\hline 
VIN [V] & $\sigma$VIN [V]& VOUT [V]& $\sigma$VOUT [V]& AV & $\sigma$AV \\ 
\hline 
• & • & • & • & • & • \\ 
\hline 
• & • & • & • & • & • \\ 
\hline 
• & • & • & • & • & • \\ 
\hline 
• & • & • & • & • & • \\ 
\hline 
• & • & • & • & • & • \\ 
\hline 
• & • & • & • & • & • \\ 
\hline 
• & • & • & • & • & • \\ 
\hline 
• & • & • & • & • & • \\ 
\hline 
• & • & • & • & • & • \\ 
\hline 
• & • & • & • & • & • \\ 
\hline 
• & • & • & • & • & • \\ 
\hline 
• & • & • & • & • & • \\ 
\hline 
• & • & • & • & • & • \\ 
\hline 
• & • & • & • & • & • \\ 
\hline 
\end{tabular}
\caption{Misure di tensione e calcolo del guadagno(?!).}
\end{table}
Inversione di fase del segnale in uscita. Valore del guadagno per piccoli segnali. Il guadagno è costante? Quando inizia il fenomeno di clipping (fare screen)?

\subsection{Impedenza di ingresso del circuito}
Come impedenza in ingresso del circuito ci aspettiamo $R_{IN}=R_1//R_2//(h_{ie}+h_{fe}R_E)$.
Abbiamo misurato la tensione in uscita del circuito in Fig.1, $V_{OUT,1}= \pm V$, e successivamente abbiamo inserito una resistenza, $R_S= \pm k\Omega$, fra il generatore e $C_{IN}$, misurando poi $V_{OUT,2}= \pm V$. Utilizzando la formula $\frac{R_S}{R_{IN}}=\frac{V_{OUT,1}}{V_{OUT,2}}-1$ ci è stato possibile ricavare il valore dell'impedenza in ingresso, $R_{IN}= \pm k\Omega$. (in accordo?!).

\subsection{Impedenza di uscita del circuito}
Come impedenza di uscita del circuito ci aspettiamo $R_{OUT}= R_C$. Come in precedenza abbiamo effettuato due misure di tensione: la prima con il circuito di partenza, $V_{OUT,1}= \pm V$, la seconda è stata presa misurata dopo aver inserito tra l'uscita e la massa una resistenza di carico R_L, $V_{OUT,2}= \pm V$. Grazie alla formula $\frac{R_{OUT}}{R_{L}}=\frac{V_{OUT,1}}{V_{OUT,2}}-1$ abbiamo ottenuto $R_{OUT}= \pm k\Omega$.

\section{Risposta in frequenza}
Abbiamo misurato la risposta in frequenza del circuito variando la frequenza da 10 Hz a 1MHz, con una tensione in ingresso $V_{IN,pp}= \pm V$. Quest'ultima è stata controllata più volte durante l'esperienza per far si che rimanesse costante durante la presa dati. Nella tabella sottostante sono riportate le misure effettuate.

\begin{table}[h]
\centering
\begin{tabular}{|c|c|c|c|}
\hline 
f [kHz] & $\sigma f [kHz]$ & $V_{OUT} [V]$ & $\sigma V_{OUT} [V]$ \\ 
\hline 
• & • & • & • \\ 
\hline 
• & • & • & • \\ 
\hline 
• & • & • & • \\ 
\hline 
• & • & • & • \\ 
\hline 
• & • & • & • \\ 
\hline 
• & • & • & • \\ 
\hline 
• & • & • & • \\ 
\hline 
• & • & • & • \\ 
\hline 
• & • & • & • \\ 
\hline 
• & • & • & • \\ 
\hline 
• & • & • & • \\ 
\hline 
• & • & • & • \\ 
\hline 
• & • & • & • \\ 
\hline 
• & • & • & • \\ 
\hline 
• & • & • & • \\ 
\hline 
• & • & • & • \\ 
\hline 
• & • & • & • \\ 
\hline 
• & • & • & • \\ 
\hline 
• & • & • & • \\ 
\hline 
• & • & • & • \\ 
\hline 
• & • & • & • \\ 
\hline 
• & • & • & • \\ 
\hline 
• & • & • & • \\ 
\hline 
• & • & • & • \\ 
\hline 
• & • & • & • \\ 
\hline 
• & • & • & • \\ 
\hline 
• & • & • & • \\ 
\hline 
• & • & • & • \\ 
\hline 
• & • & • & • \\ 
\hline 
\end{tabular} 
\caption{Risposta in frequenza del circuito Common Emitter.}
\end{table}
le misure effettuate le abbiamo poi riportate in un diagramma di Bode (fit).
Si è poi eseguita una misura diretta della frequenza di taglio del circuito (superiore e inferiore).

\section{Aumento del guadagno}
In questaultima parte si è inserita la resistenza $R_{es}= \pm k\Omega$ e si è misurato il nuovo guadagno a frequenza fissa, $f= \pm Hz$, utilizzando lo stesso metodo e la stessa formula sopra citati. I valori ottenuti sono riportati nella seguente tabella.

\begin{table}[h]
\centering
\begin{tabular}{|c|c|c|c|c|c|}
\hline 
$V_{IN}$ & $\sigma V_{IN}$ & $V_{OUT}$ & $\sigma V_{OUT}$ & $A_V$ & $\sigma A_V$ \\ 
\hline 
• & • & • & • & • & • \\ 
\hline 
• & • & • & • & • & • \\ 
\hline 
• & • & • & • & • & • \\ 
\hline 
• & • & • & • & • & • \\ 
\hline 
• & • & • & • & • & • \\ 
\hline 
• & • & • & • & • & • \\ 
\hline 
• & • & • & • & • & • \\ 
\hline 
• & • & • & • & • & • \\ 
\hline 
• & • & • & • & • & • \\ 
\hline 
\end{tabular}
\caption{Guadagno per piccoli segnali?!.}
\end{table}

Il guadagno atteso per piccoli segnali è $A_V=-\frac{R_C}{Z_E+h_{ie}/h_{fe}}\approx\frac{R_C}{Z_E}$. Confronto con quello misurato.

\end{document}


