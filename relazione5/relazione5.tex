\documentclass[10pt,a4paper]{article}
\usepackage[utf8]{inputenc}
\usepackage[italian]{babel}
\usepackage{amsmath}
\usepackage{amsfonts}
\usepackage{amssymb}
\usepackage{graphicx}
\usepackage[left=2cm,right=2cm,top=2cm,bottom=2cm]{geometry}
\newcommand{\rem}[1]{[\emph{#1}]}

\author{Gruppo AC \\ Belliardo Federico, Franchi Giulia, Mazzoncini Francesco}
\title{Esercitazione N.5: Transistor JFET.}
\begin{document}

\maketitle

\section{Scopo e strumentazione}
Studiare le caratteristiche e realizzare un amplificatore con il JFET a canale N 2N3819.

\section{Studio funzionamento del JFET}
\paragraph{Montaggio e ossevazioni qualitative.}
E' stato montato il circuito in fig. \ref{circuito1}, con  $R_1 = 0.994\pm0.008 \, k\Omega$, $V_1 = 15.11\pm0.08 \, V$ e $V_2 = -15.01 \pm 0.08 \, V$ \footnote{Misure eseguite con il multimetro digitale}. Le due sorgenti di tensione DC sono state ottenute dalle due boccole del generatore in dotazione. Le resistenze massime e minime del potenziometro (indicato con $R_2$) sono:  $R_{max} = 1.95 \pm 0.01 \, k\Omega$ e $R_{min} = 0.3 \pm 0.3 \, \Omega$

\begin{figure}
\centering
\includegraphics[scale=0.4]{circuito1.png}
\caption{Schema di amplificatore con JFET in corrente continua.\label{circuito1}}
\end{figure}

Variando la resistenza del potenziometro (partitore di tensione) cambia la tensione di \emph{gate} ($V_{GS}$), dunque il JFET entra in conduzione solamente quando si supera la tensione $V_{GS} > V_{P}$ (tensione di \emph{pinch-off}), quando ciò succede si accende il led. Qualitativamente stimiamo: $V_P \sim 3\,V$. 

\paragraph{Misura della corrente $I_D$ in funzione di $V_{GS}$.}
Si sono prese misure della tensione $V_{GS}$ e di $V_{R1}$ (caduta di potenziale ai capi di $R_1$) utilizzando il multimetro digitale \footnote{Abbiamo evitato l'uso dell'oscilloscopio perchè le nostre misure non fossero affette dall'errore sistematico del $3\%$}, da $V_{R1}$ si è ricavata $I_D = \frac{V_{R1}}{R_1}$. Nella tabella \ref{correnteId} e in fig. \ref{correnteIdVgs} sono riporati i dati presi.\\
Gli errori delle misure di tensione nei grafici e nella tabella sono calcolati come specificato nel manuale del multimetro.

\begin{table}[!htb]\centering
\begin{tabular}{|c|c|c|c|c|c|}
\hline
$ V_{R1} (V)$ & $ \sigma V_{R1} (V) $ & $V_{GS} (V) $ & $\sigma V_{GS} (V)$ & $I_D (mA)$ & $\sigma I_D (mA)$\\ 
\hline
0.013 & 0.001 & -3.27 & 0.02 & 0.013 & 0.001\\
0.078 & 0.001 & -3.13 & 0.02 & 0.079 & 0.001\\
0.264 & 0.002 & -2.94 & 0.02 & 0.266 & 0.003\\
0.462 & 0.003 & -2.81 & 0.02 & 0.465 & 0.005\\
1.02 & 0.01 & -2.51 & 0.02 & 1.03 & 0.01\\
1.69 & 0.01 & -2.23 & 0.01 & 1.70 & 0.02\\
2.94 & 0.02 & -1.81 & 0.01 & 2.96 & 0.03\\
4.34 & 0.02 & -1.37 & 0.01 & 4.37 & 0.04\\
6.22 & 0.03 & -0.872 & 0.004 & 6.26 & 0.06\\
8.01 & 0.04 & -0.413 & 0.002 & 8.06 & 0.08\\
9.36 & 0.05 & -0.037 & 0.001 & 9.42 & 0.09\\
\hline
\end{tabular}
\caption{Dati di corrente $I_D$ e di tensione $V_{GS}$, $V_{R1}$}
\label{correnteId}
\end{table}

La retta di carico è: $V_1 - R_1 I_D-V_{\gamma}-V_{DS} = 0$, dove $V_{\gamma} \sim 1.8 \, V$ è la caduta di tensione sul led rosso (caratteristica del led). Questa retta di carico è valida quando il led è acceso cioè quando vi è una corrente $I_D$: sono in zona ohmica o di saturazione, mentre $V_{DS} = V_1$ è la retta di carico quando sono in zona di interdizione.\\

\begin{figure}
\centering
\includegraphics[scale=0.4]{char2.png}
\caption{Curve caratteristiche del JFET dal datasheet.\label{curveCaratteristiche}}
\end{figure}

\rem{Aggiungere retta di carico}
La fig. \ref{curveCaratteristiche} riporta un'immagine delle curve caratteristiche del JFET nel caso in cui la tensione di \emph{pinch-off} sia $V_P = -3.0\,V$, sul quale è riportata la retta di carico. Si vede che per i valori delle tensioni $V_{DS}$ esplorati (calcolati dalla retta di carico e riportati nella tabella \ref{correnteId} siamo sempre in zona di saturazione. 

E' stato eseguito un fit di una funzione parabolica ($I_D = K_P (V_{GS} - V_P)^2$), considerando solamente i dati attorno alla tensione di \emph{pinch-off}, cioè in una regione in cui ci aspettiamo valga il comportamento ideale. 

Per il fit numerico si è utilizzata la funzione \emph{curvefit} della libreria \emph{pylab} con l'opzione \emph{$absolute\,sigma = "true"$}, poichè abbiamo considerando gli errori come statistici. Riportiamo il grafico in figura \ref{correnteIdVgs} e di seguito parametri fittati con la relativa matrice di covarianza: $K_P = 1.30 \pm 0.05 \, \frac{mA}{V^2}$, $V_P = -3.39 \pm 0.02 \, V$,  $ \Sigma_{ij} = \left( \begin{array}{cc}
3.46 \cdot 10^{-3} & 8.40 \cdot 10^{-4} \\ 
8.40 \cdot 10^{-4} & 2.69 \cdot 10^{-4}\\
\end{array} \right)$. Con un $\chi^2/ndof = 3/4$.

Il punto del grafico per cui $V_{GS} \sim 0\,V$ corrisponde alla corrente $I_{DSS} = 9.5 \pm 0.2\, mA$ \footnote{L'errore su $I_{DSS}$ è la semidispersione dell'intervallo massimo in cui è ragionevole si trovi $V_{GS} \sim 0\,V$.}, mentre $V_P \sim -3.3\,V$ (tensione a $I_D \sim 0 \, mA$), entrambe stimate dal grafico.

Alternativamente si possono utilizzare le informazioni del fit: $I_{DSS} = K_P V_{P}^2 = 15.0 \pm 0.3 \, mA$. I due valori non sono compatibili, perchè il fit esguito non può essere estrapolato fino a tensioni prossime allo zero.


\begin{figure}
\centering
\includegraphics[scale=0.5]{parabolaTutti.png}
\caption{Corrente di drain misurata in funzione della tensione $V_{GS}$.\label{tuttiIdati}}
\end{figure}

\begin{figure}
\centering
\includegraphics[scale=0.5]{parabolaFit.png}
\caption{Fit parabolico intorno alla tensione di pinch off.\label{correnteIdVgs}}
\end{figure}

Il valore di $V_P$ è molto variabile per costruzione, ma il valore misurato è compatibile con quello tipico indicato nel \emph{datasheet}: $V_{P, datasheet} = -3 \, V$. Per $I_{DSS}$ sono riportati possibili valori tra $2 \, mA$ e $20 \, mA$, entrambi i valori ottenuti sono compatibili.

\section{Montaggio amplificatore}

\paragraph{Stima della tensione $V_P$ e della corrente $I_{DSS}$.}
Si è montato il circuito in fig. \ref{circuito2}, con i componenti: $R_1 = 0.994\pm0.008 \, k\Omega $, $R_2 = 1.95\pm0.02 \, k \Omega $, $R_3 = 4.66 \pm 0.04 \, M \Omega$ e $C_1 = 99\pm4 \, nF$ e $V_1 = 15.01\pm0.08 \, V$. Si è regolato il potenziometro in modo che la corrente di quiescenza fosse la metà di $I_{DSS}$, il valore misurato di $V_{R1} = 4.49 \pm 0.03 \, V$, dal quale si ottiene: $I_D = 4.52\pm0.04\,mA$. La resistenza a cui si osserva ciò è: $R_{part} = 230\pm2 \Omega$ (è lasciata costante e sarà usata successivamente). Si è misurata la tensione $V_{GS} = 0.972 \pm 0.005$. Dalla formula \footnote{In questa formula e nelle seguenti il valori di $I_{DSS}$ è quello stimato dal grafico, mentre $V_P$ è quello ottenuto dal fit} $V_{GS} = V_{P} \left( 1 - \sqrt{\frac{I_D}{I_{DSS}}} \right)$ (valida in zona di saturazione), ricaviamo il valore atteso per $V_{GS}$ cioè: $V_{GS} = -1.05\pm0.01 \, V$.\\ 
Da questi dati si può anche dare una stima della tranconduttanza: $g_m = \frac{i_D}{v_{GS}} = \frac{2I_{DSS}}{\vert V_P \vert} \sqrt{\frac{I_D}{I_{DSS}}} = 3.87\pm0.05\,mS$. 

\begin{figure}
\centering
\includegraphics[scale=0.4]{circuito2.png}
\caption{Schema di JFET in corrente continua.\label{circuito2}}
\end{figure}

\section{Misure a frequenza fissa}
Tutte le misure di questa sezione sono prese usando una frequenza fissa di $f_0 = 1.00\pm0.01\, kHz$. L'ingresso del circuito in entrambi i casi è al gate.
\paragraph{Circuito \emph{common source}.}
Si sono prese le misure si tensione in uscita dal \emph{drain}. I dati sono riportati nella tabella \ref{tabellaCommonSource}.

\begin{table}[!htb]\centering
\begin{tabular}{|c|c|c|c|c|c|}
\hline
$V_{IN} (V)$ & $\sigma V_{IN} (V)$ & $V_{OUT} (V)$ & $\sigma V_{OUT} (V)$ & $A_V$ & $\sigma A_V$\\
\hline
0.109 & 0.001 & 0.208 & 0.001 & -1.91 & 0.02\\
0.220 & 0.002 & 0.424 & 0.001 & -1.93 & 0.02\\
0.428 & 0.002 & 0.816 & 0.001 & -1.907 & 0.009\\
0.768 & 0.004 & 1.510 & 0.002 & -1.97 & 0.01\\
0.936 & 0.002 & 1.840 & 0.002 & -1.966 & 0.005\\
1.300 & 0.004 & 2.540 & 0.002 & -1.954 & 0.006\\
\hline
\end{tabular}
\caption{Guadagno JFET in \emph{common source}.}
\label{tabellaCommonSource}
\end{table}

Trascuarando la corrente che scorre nel \emph{gate} abbiamo le due equazioni per piccoli segnali: $i_D = g_m v_{gs} = \frac{v_S}{R_{part}}$ e $i_D = g_m v_{gs} = -\frac{v_D}{R_1}$, da queste si ottiene: $A_V = -\frac{v_D}{v_G} = - \frac{R_1 g_m}{1+R_{part} g_m} = -2.01\pm0.02$. \\
Come si vede dalla tabella per gli intervalli di tensione per cui si sono prese le misure l'amplificazione rimane circa costante e il suo valore medio è: $A_V = -1.938\pm0.005$. \\
Si è iniziato ad avere clipping superiore per $V_{clipping, sup} = 5.92\pm0.04 \, V$. Abbiamo impostato l'oscilloscopio in DC e si è osservato che il \emph{clipping} taglia il segnale a $15\,V$ che è la massima tensione erogabile (tensione di alimentazione).\\
\rem{spiegare perchè non c'è clipping inferiore}
Si osserva inversione del segnale, come si vede dalla formula del guadagno in cui compare un segno meno.\\

\paragraph{Circuito \emph{source follower}.}
Nella tabella \ref{tabellaSourceFollower} sono riportati i dati prendendo come uscita il source, si sono ripetute le stesse misure e analisi.

\begin{table}[!htb]\centering
\begin{tabular}{|c|c|c|c|c|c|}
\hline
$V_{IN} (V)$ & $\sigma V_{IN} (V)$ & $V_{OUT} (V)$ & $\sigma V_{OUT} (V)$ & $A_V$ & $\sigma A_V$\\
\hline
0.114 & 0.001 & 0.056 & 0.001 & 0.49 & 0.01\\
0.174 & 0.002 & 0.084 & 0.001 & 0.483 & 0.008\\
0.254 & 0.002 & 0.126 & 0.001 & 0.496 & 0.006\\
0.346 & 0.002 & 0.172 & 0.002 & 0.497 & 0.006\\
0.444 & 0.004 & 0.220 & 0.002 & 0.495 & 0.006\\
0.552 & 0.004 & 0.274 & 0.002 & 0.496 & 0.005\\
\hline
\end{tabular}
\caption{Guadagno JFET in \emph{source follower}.}
\label{tabellaSourceFollower}
\end{table}

Dalle stesse equazioni della sezione precedente otteniamo la relazione: $A_V = \frac{R_{part} g_m}{1+R_{part} g_m}$, dalla quale si può stimare teoricamente il guadagno atteso come: $A_V = 0.478\pm0.004$. \\
La media delle misure è $A_V = 0.493\pm0.003m$. I due valori non sono in accordo entro l'errore sperimentale, poich nella propagazione sono stati trascurati gli errori sistematici dell'oscilloscopio al $3\%$. In questo caso non si ha inversione, come si può vedere dal segno positivo del guadagno atteso.\\
Si osserva clipping inferiore alla tensione: $V_{clipping, inf} = 6.24\pm0.04\,V $. Il segnale in uscita in questo caso è saturato a $\sim -15\,V$, che è la tensione dell'alimentazione inferiore.\\
\rem{Spiegare perchè non si ha clipping superiore}

\rem{vedere se mettere l'immagine del modello a piccoli segnali, soprattuto quale mettere...}

Nella formula per determinare il guadagno vediamo $g_m$ sia a numeratore che a denominatore, dunque non possiamo propagare l'errore considerandoli come indipendenti (sovrastimeremmo troppo l'errore sull'amplificazione). La propagazione statistica eseguita con le derivate parziali (di $A_V(g_m, R_1, R_{part})$) sommate in quadratura li considera come errori non indipendenti, quindi si è eseguito il calcolo in questo modo.\\

\section{Misura impedenza di ingresso}
Trascurando le impedenze tra i terminali del JFET possiamo stimare $R_{int} = \frac{1}{j \omega C} + R_3 \sim R_3 = 4.66\pm0.04 M\Omega$. Per eseguire la misura si sono misurate le uscite con e senza resistenza $R_{S} = 5.20 \pm 0.05 \, M\Omega$ posta in serie al generatore di funzioni. La resistenza in ingresso misurata si ottiene dalla formula del partitore di tensione, $\frac{R_S}{R_IN} = \frac {V_1}{V_2} - 1$ (dove $V_1$ è la tensione misurata senza resistenza $R_S$). Si sono eseguite le misure per le frequenze $f_1 = 1 kHz$ e $f_2 = 10 kHz$.

In tabella sono anche riportate le resistenze attese calcolate teoricamente alle due frequenze:\\

\begin{tabular}{|c|c|c|c|c|}
\hline 
• & $V_1 (V)$ & $V_2$ & $R_{IN, mis} (M\Omega)$ & $R_{IN, att} (M\Omega)$ \\ 
\hline
$1 kHz$ & $1.43 \pm 0.01$ & $0.648 \pm 0.004$ & $4.31 \pm 0.08$ & • \\ 
\hline 
$10 kHz$ & $1.43 \pm 0.01$ & $0.206 \pm 0.002$ & • & • \\ 
\hline 
\end{tabular}\\
Si nota che 
L'impedenza misurata sperimentalmente è minore di quella calcolata teoricamente a causa delle impendenze delle capacità tra i terminali del JFET, che sono poste in parallelo alla resistenzea $R_3$.

\section{Aumento del guadagno}
In questa sezione si è mantenta costante la frequenza di lavoro ($f_0 = 1.00\pm0.01\, kHz$) e variando il potenziometro si sono effettuate diverse misure di tensione in uscita. \rem{abbiamo dovuto verificare che l'ingresso fosse costante?}. Il valore massimo del guadagno è risultato essere quello per cui la resistenza $R_S$ era minore (teoricamente nulla) ($R_{S, min} = 0.3 \pm 0.3 \Omega$). 
Il valore teorico del guadagno con questa resistenza è: $A_V = \pm $, che non è compatibile con il valore misurato.

\end{document}
