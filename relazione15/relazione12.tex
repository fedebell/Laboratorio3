\documentclass[10pt,a4paper]{article}
\usepackage[utf8]{inputenc}
\usepackage[italian]{babel}
\usepackage{amsmath}
\usepackage{amsfonts}
\usepackage{amssymb}
\usepackage{graphicx}
\usepackage{gensymb}
\usepackage[left=2cm,right=2cm,top=2cm,bottom=2cm]{geometry}
\newcommand{\rem}[1]{[\emph{#1}]}

\author{Gruppo BN \\Lisa Bedini,  Federico Belliardo, Marco Costa}
\title{Esperienza 15: Misura della costante di Boltzmann}

\begin{document}
\maketitle
\section{Scopo dell'esperienza}
Misurare la costante di Bolzmann dal rumore termico (Johnson-Nyquist) di una resistenza a temperatura nota, grazie ad un amplificator, un filtro passa-banda e un convertitore RMS.

\section{Materiale a disposizione}
\begin{itemize}
\item INA114 Amplificatore
\item AD708: OpAmp integrati
\item AD736: RMS converter


\section{Montaggio del circuito}
\subsection{Power filter}
La tensione di alimentazione e la massa sono state filtrate mediante circuiti passa-basso prima di fornirle agli integrati in modo da ridurne le fluttuazioni:




\section{Misura della costante di Boltzmann}


\section{Conclusioni}

\end{document}







