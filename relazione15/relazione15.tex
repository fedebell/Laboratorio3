\documentclass[10pt,a4paper]{article}
\usepackage[utf8]{inputenc}
\usepackage[italian]{babel}
\usepackage{amsmath}
\usepackage{amsfonts}
\usepackage{amssymb}
\usepackage{graphicx}
\usepackage{gensymb}
\usepackage[left=2cm,right=2cm,top=2cm,bottom=2cm]{geometry}
\newcommand{\rem}[1]{[\emph{#1}]}

\author{Gruppo BN \\Lisa Bedini,  Federico Belliardo, Marco Costa}
\title{Esperienza 15: Misura della costante di Boltzmann}

\begin{document}
\maketitle
\section{Scopo dell'esperienza}
Misurare la costante di Bolzmann dal rumore termico (Johnson-Nyquist) di una resistenza a temperatura nota, grazie ad un amplificator, un filtro passa-banda e un convertitore RMS.

\section{Materiale a disposizione}
\begin{itemize}
\item INA114 Amplificatore
\item AD708: OpAmp integrati
\item AD736: RMS converter
\end{itemize}

\section{Montaggio del circuito}
\subsection{Power filter}
La tensione di alimentazione e la massa sono state filtrate mediante circuiti passa-basso prima di fornirle agli integrati in modo da ridurne le fluttuazioni:\\

\begin{figure}[!htb]
\centering
\includegraphics[scale=0.5]{powerfilter.png}
\caption{Filtri passa basso per le alimentazioni.\label{power}}
\end{figure}

Si è montato anche il filtro per la tensione negativa anche se non è mai stata usata.\\

\subsection{Preamplificatore}
Il primo stadio di amplificazione è stato realizzato con un Precision instrumentation amplifier avente un guadagno atteso: $G_1 = 1+\frac{50 k\Omega}{1 k\Omega} = 51$ il secondo stadi è un semplice amplificatore invertente con guadagno teorico: $G_2 = \frac{68 k\Omega}{4.7 k\Omega} = 14.5$. Il valore ateso dell'amplificazione è: $G = 740$. Non sono stati propagati gli errori sulle resistenze. Si nota che anche volendo farlo non conosciamo la precisione sulla calibrazione della resistenza interna dall'INA114.\\

\begin{figure}[!htb]
\centering
\includegraphics[scale=0.3]{preamp.png}
\caption{Preamplificatore realizzato con opAmp e INA114.\label{preamp}}
\end{figure}

Il circuito è stato analizzato fornendo in ingresso con il generatore di funzioni un onda sinusoidale di piccola ampiezza ($V_0 = $) e se ne è costruito il diagramma di Bode, come visualizzato in figura \fig{bode}. Esso si comporta come u filtro passa basso con una frequenza di taglio $f = $. Il segnale in uscita appare molto rumoroso.\\

\subsection{Filtro passa-banda e post amplificatore}
Si sono realizzati il filtro passa-banda e il post amplificatore come mostrato nelle figure.


\begin{figure}[!htb]
\centering
\includegraphics[scale=0.3]{banda.png}
\caption{Filtro passa-banda realizzato con opAmp.\label{banda}}
\end{figure}


\begin{figure}[!htb]
\centering
\includegraphics[scale=0.3]{postamp.png}
\caption{Post amplificatore realizzato con opAmp.\label{postamp}}
\end{figure}

L'analisi teorica del circuito passa-banda indica che questo ha una amplificazione massima $A_{banda} = \frac{R_3}{2 R_1} = 8.7$ in corrispondenza della frequenza $f_t = \frac{1}{2 \pi C} \sqrt{\frac{\frac{1}{R_1} + \frac{1}{R_2}}{R_3}} = 6.4 kHz$. La larghezza i banda prevista è: $\Delta f = \frac{1}{\pi R_3 C}$, agli estremi della quale l'amplificazione è di $3 dB$ inferiore a quella di picco. Il post amplificatore ha invece un guadagno $A_{post} = 1 + \frac{R_2}{R_1} = 34$, come è ben noto. Il valore a centro banda atteso sarebbe dunque $A_{0} = 269$.\\

%TODO --> Finire....
E' stato realizzato un plot di Bode per questa parte di circuito e si sono confrontate le previsioni teoriche...

Non è stato eseguito il fit di questo Bode.

\subsection{Convertitore RMS}
L'ultima parte del circuito ha lo scopo di trasformare un segnale alternato in un segnale continuo restituendone il valore quadratico medio ed è stato realizzato con un apposito integrato riportato in figura:


\begin{figure}[!htb]
\centering
\includegraphics[scale=0.3]{rms.png}
\caption{Circuito RMS realizzato con appostito integrato.\label{rms}}
\end{figure}

\section{Misura della costante di Boltzmann}
%TODO ---> FINIRE
Il rumore ($V_{rms})$ totale è determinato dalla somma in quadratura dei rumori associati alla resistenza e quelli dovuti all'amplificatore, parametrizzati come rumori in serie e in parallelo all'amplificatore. Dalla teoria ci aspettiamo dunque: $V_{rms} = V_0 \sqrt{1+\frac{R}{R_t}+(\frac{R}{R_n})^2}$ dove:
\begin{itemize}
\item $V_0$ è il rumore in uscita a resistenza nulla (dato solo dal rumore in serie all'amplificatore)
\item $R_t = \frac{V_0^2}{4 k_b T A_0 \Delta f}$ è la resistenza equivalente del rumore in serie dell'amplificatore riferito all'ingresso
\end{itemize}

%Inserire parte relativa al fit...



\section{Conclusioni}

\end{document}







