\documentclass[10pt,a4paper]{article}
\usepackage[utf8]{inputenc}
\usepackage[italian]{babel}
\usepackage{amsmath}
\usepackage{amsfonts}
\usepackage{amssymb}
\usepackage{graphicx}
\usepackage{gensymb}
\usepackage[left=2cm,right=2cm,top=2cm,bottom=2cm]{geometry}
\newcommand{\rem}[1]{[\emph{#1}]}

\author{Gruppo BN \\ Lisa Bedini, Federico Belliardo, Marco Costa}
\title{Esperienza 14: Misura della costante di assorbimento del Mylar usando un amplificatore lock-in}

\begin{document}
\maketitle

\section{Scopo dell'esperienza}
Lo scopo dell'esperienza è di effettuare una misura di assorbimento della luce di uno spessore di Mylar utilizzando un amplificatore sincrono sensibile alla fase (lock-in).

\section{Materiale occorrente}
\begin{itemize}
\item TL082: JFET dual Op-Amp
\item TL081: JFET Op-Amp
\item SN7400 Quad NAND gates
\item DG441: Quad CMOS analog switch
\item 2N1711, BC182: NPN transistor
\item LED rosso
\item Fotodiodo
\end{itemize}

\section{Schema a blocchi e metodo di misura}

\subsection{Amplificatore di potenza e preamplificatore}

\subsection{Sfasatore di 90$\circ$ e sfasatore a fase variabile}

\subsection{Squadratore e campionatore}

\subsection{Amplificatore differenziale e mediatore}

\section{Conclusioni}


\end{document}
