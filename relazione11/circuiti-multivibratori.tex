\documentclass[10pt,a4paper]{article}
\usepackage[utf8]{inputenc}
\usepackage[italian]{babel}
\usepackage{amsmath}
\usepackage{amsfonts}
\usepackage{amssymb}
\usepackage{graphicx}
\usepackage[left=2cm,right=2cm,top=2cm,bottom=2cm]{geometry}
\newcommand{\rem}[1]{[\emph{#1}]}

\author{Gruppo BN \\ Federico Belliardo, Marco Costa, Lisa Bedini}
\title{Semplici circuiti logici e Multivibratori}
\begin{document}

\maketitle
\section{Scopo dell'esperienza}
Nella prima parte dell'esperienza ci si propone di montare e verificare il funzionamento di semplici circuiti logici (AND, OR, XOR e sommatore a un bit) utilizzando solo porte NAND. Successivamente saranno montati un circuito Multivibratore monostabile e astabile per verificare la dipendenza lineare tra tempo di durata dell'impulso in uscita e la resistenza presente. Infine questi ultimi due circuiti verranno posti in serie per formare un generatore di onda quadra, per studiare la dipendenza tra le resistenze usate e il \emph{duty cycle}.\\

\section{Materiale occorrente}
\begin{itemize}
\item 2 circuiti integrati SN7400 Quad-NAND Gate;
\item DIP Switch a 4 interruttori;
\item Diodo 1N4148;
\item 2 diodi LED;
\end{itemize}
Disponiamo inoltre del circuito pulsatore montato nella precedente esperienza, costituito da un Arduino Nano e da un octal buffer/driver SN74LS244.
Tutte le resistenze, i condensatori e la tensione di alimentazione sono stati misurati con il multimetro digitale, quindi l'errore è stato propagato secondo le specifiche nel manuale. I tempi e le restanti tensioni sono state misurate con i cursori dell'oscilloscopio: l'errore sui tempi è dato dalla risoluzione dei cursori stessi mentre quello sulle tensioni è stato propagato considerando sia l'errore sul posizionamento dei cursori sia l'errore sistematico del $3%$.\\

\section{Semplici circuiti logici}
\subparagraph{Verifica porta NAND}
Abbiamo montato il circuito in figura \ref{fig:NAND}, con una tensione di alimentazione pari a $V_{CC}= 4.85\pm0.03\,\text{V}$, e ne abbiamo verificato il funzionamento prima tramite il diodo LED poi tramite l'oscilloscopio. Si sono usati due interruttori e una resistenza di pull-up $R_1=327\pm3\,\Omega$ per mantenere l'input a livello alto anche nel caso di interruttori aperti\footnote{Sapendo che per ottenere uno zero nella logica TTL bisogna collegare materialmente a 0 l'ingresso dato che di default l'ingresso è alto.}. In tabella \ref{tab:NAND} si possono vedere i valori di output attesi, 1 corrisponde al livello alto mentre lo 0 corrisponde al livello basso. Si nota che il LED è spento nel caso di $I_1=I_2=0$ mentre è acceso in tutti gli altri casi. La verifica con l'oscilloscopio si effettua inserendo come input il circuito pulsatore di Arduino\footnote{Abbiamo usato una frequenza di circa 1kHz.}, in questo modo si possono visualizzare tutti gli stati con l'oscilloscopio collegando ad un canale alternativamente l'ingresso $I_1$ e $I_2$., si veda l'immagine \ref{osc:NAND}. Abbiamo usato la traccia di output per il trigger. Nuovamente troviamo la corretta tabella di verità per il NAND.\\


\begin{figure}[!htb]
  \centering
  \includegraphics[scale=0.5]{NAND.png}
\caption{Schema circuitale della porta NAND.\label{fig:NAND}}
\end{figure}

\begin{table}[!htb]
\centering
\begin{tabular}{|c|c|c|}
\hline 
$I_1$ & $I_2$ & $O$ \\
\hline
 1 &  0 & 1\\ 
 
 1 &  1 & 0\\ 

 0 &  1 & 1\\ 
 
 0 &  0 & 1\\ 
\hline 
\end{tabular} 
\caption{Tabella di verità della porta NAND.\label{tab:NAND}}
\end{table}

\begin{figure}[!htb]
  \centering
  \includegraphics[scale=1]{nand1.png}\includegraphics[scale=1]{nand2.png}
\caption{Schermate dell'oscilloscopio, in canale 1 c'è l'output e in canale 2 l'input.\label{osc:NAND}}
\end{figure}


\subparagraph{Circuito AND}
E' stato realizzato il circuito in figura \ref{fig:AND}, anche in questo caso si è visualizzato l'output sull'oscilloscopio (figura \ref{osc:AND}), triggerando sul segnale in uscita. Si nota che l'andamento è quello previsto dalla tabella \ref{tab:AND}, infatti nelle immagini fornite dall'oscilloscopio si nota che soltanto quando entrambi gli ingressi sono a livello alto, anche l'uscita è alta.\\

\begin{figure}[!htb]
  \centering
  \includegraphics[scale=0.5]{AND.png}
\caption{Schema del circuito AND.\label{fig:AND}}
\end{figure}

\begin{table}[!htb]
\centering
\begin{tabular}{|c|c|c|}
\hline 
$I_1$ & $I_2$ & $O$ \\
\hline
 1 &  0 & 0\\ 

 1 &  1 & 1\\ 

 0 &  1 & 0\\ 

 0 &  0 & 0\\ 
\hline 
\end{tabular} 
\caption{Tabella di verità del circuito AND.\label{tab:AND}}
\end{table}

\begin{figure}[!htb]
  \centering
  \includegraphics[scale=0.75]{and1.png}\includegraphics[scale=0.75]{and2.png}
\caption{Schermate dell'oscilloscopio, in canale 1 c'è l'output e in canale 2 l'input.\label{osc:AND}}
\end{figure}

\subparagraph{Circuito OR}

E' stato montato il circuito in figura \ref{fig:OR}. In tabella \ref{tab:OR} è stata rappresentata la tabella di verità e in figura \ref{osc:OR} si può osservare l'andamento dell'output. Si nota che l'uscita è a livello basso quando entrambi gli ingressi sono a 0. Anche in questo caso abbiamo triggerato sull'output e i risultati sono in accordo con la tabella di verità di un OR.


\begin{center}
\begin{minipage}{0.3\textwidth}
\begin{table}[!htb]
\begin{tabular}{|c|c|c|}
\hline 
$I_1$ & $I_2$ & $O$ \\
\hline
 1 &  0 & 1\\ 

 1 &  1 & 1\\ 
 
 0 &  1 & 1\\ 
 
 0 &  0 & 0\\ 
\hline 
\end{tabular} 
\caption{Tabella di verità del circuito OR.\label{tab:OR}}
\end{table}
\end{minipage}\qquad
\includegraphics[width=.4\textwidth]{OR.png}
\end{center}

\begin{figure}[!htb]
  \centering
  \includegraphics[scale=0.5]{OR.png}
\caption{Schema del circuito OR.\label{fig:OR}}
\end{figure}

\begin{table}[!htb]
\begin{tabular}{|c|c|c|}
\hline 
$I_1$ & $I_2$ & $O$ \\
\hline
 1 &  0 & 1\\ 

 1 &  1 & 1\\ 
 
 0 &  1 & 1\\ 
 
 0 &  0 & 0\\ 
\hline 
\end{tabular} 
\caption{Tabella di verità del circuito OR.\label{tab:OR}}
\end{table}

\begin{figure}[!htb]
  \centering
  \includegraphics[scale=0.75]{or1.png}\includegraphics[scale=0.75]{or2.png}
\caption{Schermate dell'oscilloscopio, in canale 1 c'è l'output e in canale 2 l'input.\label{osc:OR}}
\end{figure}

\subparagraph{Circuito XOR}
In questo caso abbiamo dovuto eseguire il trigger su un ingresso perchè la frequenza dell'uscita è doppia rispetto a quella dell'ingresso.


\begin{figure}[!htb]
  \centering
  \includegraphics[scale=0.5]{XOR.png}
\caption{Schema del circuito XOR.\label{fig:XOR}}
\end{figure}


\begin{table}[!htb]
\centering
\begin{tabular}{|c|c|c|}
\hline 
$I_1$ & $I_2$ & $O$ \\
\hline
 1 &  0 & 1\\ 
 
 1 &  1 & 0\\ 

 0 &  1 & 1\\ 

 0 &  0 & 0\\ 
\hline 
\end{tabular} 
\caption{Tabella di verità del circuito XOR.\label{tab:XOR}}
\end{table}

\begin{figure}[!htb]
  \centering
  \includegraphics[scale=0.75]{xor1.png}\includegraphics[scale=0.75]{xor2.png}
\caption{Schermate dell'oscilloscopio, in canale 1 c'è l'output e in canale 2 l'input.\label{osc:XOR}}
\end{figure}

\subparagraph{Circuito sommatore a un bit}
Il circuito sommatore a un bit in figura \ref{fig:sommatore} è stato montato aggiungendo al circuito XOR un NOT che preleva il segnale in uscita dal NOT tra i due segnali in ingresso e fornisce l'uscita R. Anche in questo caso abbiamo scritto la tabella di verità (tabella \ref{tab:sommatore}) e visualizzato l'output con l'oscilloscopio (figura \ref{osc:sommatore}). Abbiamo eseguito il trigger sull'uscita R e i risultati sono in accordo con quanto atteso.

%canale 2 R
%canale 1 S e ingressi
%trigger su R


\begin{figure}[!htb]
  \centering
  \includegraphics[scale=0.5]{sommatore.png}
\caption{Schema del circuito sommatore a un bit.\label{fig:sommatore}}
\end{figure}

\begin{table}[!htb]
\centering
\begin{tabular}{|c|c|c|c|}
\hline 
$I_1$ & $I_2$ & $S$ & $R$\\
\hline
 1 &  0 & 1 & 0\\ 
 
 1 &  1 & 0 & 1\\ 

 0 &  1 & 1 & 0\\ 
 
 0 &  0 & 0 & 0\\ 
\hline 
\end{tabular} 
\caption{Tabella di verità del circuito sommatore.\label{tab:sommatore}}
\end{table}

\begin{figure}[!htb]
  \centering
  \includegraphics[scale=0.75]{sommatoreRS.png}
\caption{Schermate dell'oscilloscopio, in canale 1 c'è S e in canale 2 R.\label{osc:sommatore}}
\end{figure}

\section{Multivibratore monostabile}
Abbiamo montato il circuito in figura \ref{fig:monostabile}. I componenti usati sono: $R_1= 470\pm4\,\text{\Omega}$ e $C_1= 110\pm4 \,\text{nF} $. Si è scelta come corrente di alimentazione $V_{CC}= 4.85\pm0.03\,\text{V}$ e come frequenza dell'onda quadra inviata dal generatore di funzioni $f = 5.16\pm0.05\,\text{kHz}$. Il periodo risulta quindi pari a $T=194\pm2\,\text{\mu s}$ e la durata dell'impulso in uscita $t=12.8\pm0.2\,\text{\mu s}$, ottenendo così un \emph{duty cycle} pari a $6.6\pm0.1% $.\\
Si è misurata una tensione massima di $4.3\pm0.1\,\text{V}$. Abbiamo osservato all'oscilloscopio l'andamento delle tensioni $V_{IN}$, $V_{OUT}$ e $V_{C}$, notando che quando in ingresso si ha l'impulso, NAND1 lo interpreta come un 1 quindi all'ingresso di NAND2 si ha uno 0, l'uscita è alta e il condensatore si carica tramite la resistenza. A causa della carica di $C_1$, $V_C$ diminuisce esponenzialmente fino al valore di commutazione $V_C=1.44\pm0.04\,\text{V}$ che corrisponde a $V_{IH}$ per NAND3. Raggiunto questo valore l'ingresso di NAND3 è interpretato come uno 0 quindi all'ingresso di NAND2 si hanno due 1, pertanto la sua uscita diventa bassa e la tensione $V_C$ diventa negativa. Il diodo entra quindi in interdizione e limita il valore di $V_C$ a $0.80\pm0.02\,\text{V}$. A questo punto il condensatore si carica finchè non arriva un altro impulso. In questo ciclo la carica del condensatore si conserva al variare dei valori logici perchè i tempi di commutazione dei NAND sono molto minori dei tempi di carica e scarica di $C_1$. Il tempo atteso caratteristico del circuito RC è $\tau_{att}=52\pm2\,\text{\mu s}$.
Abbiamo verificato che variando la frequenza in ingresso da circa $33 kHz$ a circa $10 kHz$ la durata dell'impulso t in uscita non cambia, mentre si ha una dipendenza di t dalla resistenza. Infatti abbiamo variato il valore della resistenza $R_1$ e misurato la durata dell'impulso, i dati sono presenti in tabella \ref{tab:monostabile}. Per convalidare l'ipotesi di linearità è stato eseguito un \emph{fit} lineare del tipo $t=aR_1+b$ ottenendo i seguenti valori: a= b= chi^2_{rid}= . Dal grafico in figura \ref{grafico:monostabile} è evidente l'andamento lineare eccetto che per gli ultimi punti con un valore della resistenza alto, per cui si ha un aumento del chi^2. Questo è probabilmente dovuto al fatto che l'andamento lineare è apprezzabile se i valori delle resistenze non si discostano troppo da quello di $R_1$, mentre per valori molto maggiori non si può più assumere la linearità a priori.

% periodo T=194\pm2 us
% t up 12.8\pm 0.2 us
%ampiezza in 0.140-4.40 V errore 0.02

%con periodo in 30 ms
%periodo 100 ms

%osservare V_IN, V_C e V_OUT con l'oscilloscopio
% vedere che variando l'impulso input, impulso output non cambia se in è minore di out
%spiegazione teorica e immagini
%cambiare R_1 e fit lineare R_1 vs durata impulso in uscita
%R2 327 ohm
%R3 669 ohm
%R4 824 ohm
%R5 984 ohm
%R6 1.183 kohm
%R7 1.454 kohm

%nand3 commuta quando V_c= 1.44 
%funzione del diodo V_C = 0.8 V
%t_up= 41.60 7pm 0.2 us
%per vedere quando nand3 commuta abbiamo misurato il potenziale vc a cui l'uscita nand 3 cambia di segno

\begin{figure}[!htb]
  \centering
  \includegraphics[scale=0.5]{monostabile.png}
\caption{Schema del circuito multivibratore monostabile.\label{fig:monostabile}}
\end{figure}

\begin{table}[!htb]
\centering
\begin{tabular}{|c|c|c|c|}
\hline 
$R_i [\Omega]$ & $\DeltaR_i [\Omega]$ & $t [\mu s]$ & $\Deltat [\mus]$\\
\hline
 327 &  3 & 25.8 & 0.2\\ 
\hline 
 470 &  4 & 41.6 & 0.2\\ 
\hline
 669 &  5 & 69.2 & 0.6\\ 
\hline
 824 &  7 & 82 & 1\\ 
\hline 
 984 &  8 & 102 & 1\\ 
\hline
 1183 &  9 & 118 & 1\\ 
\hline
 1454 &  11 & 140 & 1\\ 
\hline
\end{tabular} 
\caption{Presa dati per verificare la linearità fra $R$ e $t$.\label{tab:monostabile}}
\end{table}

%i punti che non tornano sono gli ultimi probabilmente perchè i valori delle resistenze si discostano molto da quello iniziale


\section{Multivibratore astabile}
Abbiamo montato il circuito in figura \ref{astabile}, misurando con il multimetro digitale $R_2= 989\pm8\,\text{\Omega}$, $C_2= 107\pm4 \,\text{nF} $ e $V_{CC}= 4.85\pm0.03\,\text{V}$. Si sono osservati all'oscilloscopio $V_{C,2}$ e $V_{OUT}$, quindi abbiamo misurato il periodo in uscita $T= 202\pm1\,\text{\mu s} $ e $t= 128\pm1,\text{\mu s} $ \emph{duty cycle} pari a $63.4\pm0.6%$. In questo circuito tutte le porte NAND sono usate in configurazione NOT, quindi se l'ingresso di NAND5 è alto, l'uscita di NAND6 è alta, quindi il condensatore si carica attraverso la resistenza provocando una diminuzione esponenziale della tensione $V_{C,2}$ fino al valore di commutazione $V_{IH}=1.48\pm0.04\,\text{V}$. A questo valore $V_{C,2}$ ha un brusco cambiamento fino a -1.16 V e l'uscita del NAND7 diventa alta, quindi il condensatore si scarica e $V_{C,2}$ aumenta fino al valore $1.44\pm0.04\,\text{V}$ quando si ha un altro brusco cambiamento di $V_{C,2}$. L'elemento NAND8 serve a invertire il segnale.

%t=128 \pm1 mus
%T=202\pm1 mus
%vc2 max = 4.08 V
%vc2commutazione out a = 1.48 V
%vC2 min = -1.16 V (simmetrico forse no)
% uscita out positiva quando c si carica fino a 1.44 V perchè è la tensione per cui si passa da livello alto a basso
%ampiezza out 3.52 V

%spiegazione circuito e immagini oscilloscopio
%misurare componenti
Come per il circuito multivibratore monostabile abbiamo variato il valore della resistenza per verificare la dipendenza lineare tra $R_2$ e il periodo in uscita T, quindi abbiamo eseguito un \emph{fit} lineare e ottenuto TOT.
%tabella e fit

%R 1.788 kohm
%R 2.13 kohm

%propagato errore sia su x che su y a=0.2

\begin{table}[!htb]
\centering
\begin{tabular}{|c|c|c|c|}
\hline 
$R_i [\Omega]$ & $\DeltaR_i [\Omega]$ & $t [\mus]$ & $\Deltat [\mus]$\\
\hline
 669 & 5 & 141  & 1\\ 
\hline
 824 &  7 & 170 & 1\\ 
\hline 
 989 &  8 & 202 & 1\\ 
\hline
 1183 &  9 & 243 & 1\\ 
\hline
 1454 &  11 & 298 & 2\\ 
\hline
 1788 & 14 & 354 & 2 \\
\hline
 2130 & 17 & 446 & 3\\
\end{tabular} 
\caption{Presa dati per verificare la linearità fra $R$ e $T$.\label{tab:astabile}}
\end{table}

\begin{figure}[!htb]
  \centering
  \includegraphics[scale=0.5]{astabile.png}
\caption{Schema del circuito multivibratore astabile.\label{fig:astabile}}
\end{figure}

\section{Generatore di onda quadra}
Il multivibratore astabile è stato collegato al monostabile tramite un derivatore, in modo da ottenere il generatore di onda quadra in figura \ref{fig:generatorequadra}. Si sono usati gli stessi componenti dei circuiti precedenti come $R_1$, $R_2$, $C_1$ e $C_2$ e si sono misurate $R_3=989\pm8\,\text{\Omega}$ e $C_3= 10.2\pm0.4 \,\text{nF}$. Da un'analisi qualitativa del circuito si suppone che il periodo dell'onda all'uscita del monostabile dipenda solo dalla costante tempo $\tau_2=C_2R_2$ e che la durata dell'impulso t dipenda solo dalla costante $\tau_1=C_1R_1$. Per verificare questa ipotesi abbiamo cambiato $R_1$ e $R_2$ in modo alternato, i dati sono visibili in tabella \ref{tab:generatore}. Come atteso, cambiando il valore della sola $R_1$ non si hanno variazioni significative del periodo; viceversa, modificando solo il valore di $R_2$ la durata dell'impulso non cambia.
Infine abbiamo stimato i valori delle resistenze per ottenere $T=100\,\text{\mu s}$ e \emph{duty cycle} pari al $30%$, sfruttando le relazioni lineari ottenute dai \emph{fit}: $R_1\simeq $ e $R_2 \simeq $. L'ultima misura presente in tabella \ref{tab:generatore} si riferisce a questa parte e il \emph{duty cycle} relativo risulta pari al TOT

%R2 989
%R3 985 ohm
%C3 10.22 nF

%sensibile al fronte di salita del derivatore, non ai fronti di discesa quindi il monostabile è triggerato dal fronte di salita

outM
%T= 202\pm1 mus
%t= 46\pm1 mus
% V 3.48 V

inM
%offset = 744 \pm 8 mV
%vmax=3.60 V rispetto a 0
%vmin = -1.16 \pm0.05 V
%stima tempo discesa = 7.16 \pm 0.08 mus
%stima 2 tau salita = 5.1\pm0.1

%spiegazione teorica
%immagini




\begin{figure}[!htb]
  \centering
  \includegraphics[scale=0.5]{generatorequadra.png}
\caption{Schema del generatore di onda quadra.\label{fig:generatorequadra}}
\end{figure}


%R1 385  328
%R1 = 470; R_2 = 1454; T= 290 \pm 2 mus; t=46.60\pm 0.2 mus
%R_1 = 385; R_2 = 1454; T= 290 \pm 2mus; t=36.0\pm0.2 mus
%R_1= 385; R_2= 1183; T= 238 \pm1 mus; t= 35.8\pm0.2 mus
%R_1=218; R_2=1183; T= 238\pm1 mus;	t= 16.30\pm0.1 mus
%R_1= 218; R_2= 824; T= 171\pm1 mus; t= 16.3\pm0.1
%R_1= 385; R_2= 669; T=	142\pm1 mus;	t=34.6\pm0.2
%R_1=328; R_2= 469; 	T=105\pm 1;	t=	27.6\pm0.2 mus	
%R_1	=385; R_2= 469; T=108\pm1		t=32.8

\begin{table}[!htb]
\centering
\begin{tabular}{|c|c|c|c|c|c|c|c|}
\hline 
$R_{1i} [\Omega]$ & $\DeltaR_{1i} [\Omega]$ & R_{2i} & dR_{2i} &$T [\mus]$ & $\Delta T [\mus]$&$t [\mus]$ & $\Delta t [\mus]$\\
\hline
 470 & 3 & 1454	& 11 & 290 & 2& 46.6& 0.2\\ 
\hline
385 & 3 & 1454	& 11 & 290 & 2& 36.0& 0.2\\ 
\hline 
 385 & 3 & 1183	& 9 & 238 & 1& 35.8& 0.2\\
\hline
218 & 2 & 1183	& 9 & 238 & 1& 16.3& 0.1\\ 
\hline
218 & 2 & 824	& 7 & 171 & 1& 16.3& 0.1\\ 
\hline
385 & 3 & 669	& 5 & 142 & 1& 34.6& 0.2\\ 
\hline
 385 & 3 & 469	& 3 & 108 & 1& 32.8& 0.2\\ 
 \hline
\end{tabular} 
\caption{Presa dati per verificare la dipendenza della forma d'onda in uscita dalle resistenze.\label{tab:generatore}}
\end{table}

\end{document}