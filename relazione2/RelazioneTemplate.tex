\documentclass[10pt,a4paper]{article}
\usepackage[utf8]{inputenc}
\usepackage[italian]{babel}
\usepackage{amsmath}
\usepackage{amsfonts}
\usepackage{amssymb}
\usepackage{graphicx}
\usepackage[left=2cm,right=2cm,top=2cm,bottom=2cm]{geometry}
\newcommand{\rem}[1]{[\emph{#1}]}

\author{Gruppo AC \\ Federico Belliardo, Francesco Mazzoncini, Giulia Franchi}
\title{Esercitazione N.2: Circuito RC - Filtri Passivi.}
\begin{document}

\maketitle

\section{Scopo e strumentazione}

Misurare la frequenza di un filtro passa-basso e studiare la variazione della risposta del filtro in funzione del carico applicato a valle. In seguito studiare la l'attenuazione di un filto passa-banda.

\section{Filtro passa-basso}
\paragraph{Progettazione filtro}
Si vogliono trovare i valori dei componenti resistivo e capacitivo del filtro perchè trasmetta un segnale sinusoidale di frequenza $2kHz$ e attenui il rumore a $20kHz$.

\begin{figure}[h]
\centering
\includegraphics[scale=0.4]{passabasso.png}
\caption{Schema del circuito passa-basso}
\end{figure}

Risolvendo il circuito e chiamando $r$ la resitenza di carico e $R$ la resistenza del passabasso si ottiene la seguente relazione per il modulo dell'attenuazione:
$\vert A(\omega) \vert = \frac{1}{\sqrt{(1+x)^2+(\frac{f}{f_{0}})^2}} $. Dove $f_0$ è la frequenza di taglio del filtro. Definite $f_2 = 20 kHz$ e $f_1 = 2 kHz$ le freuenze del rumore e del segnale e $x = \frac {R}{r}$ otteniamo come rapporto di attenuazione: 
$\vert \frac{A(2 kHz)}{A(20 kHz} \vert = \sqrt{\frac{{f_0}^2 (1+x)^2 + {f_2}^2}\frac{{f_0}^2 (1+x)^2 + {f_1}^2}}$

Selezionando una resistenza $R = 1 k\Omega$ del filtro molto minore del carico $r = 100 k\Omega$ otteniamo $x = 0.01$ cioè un valore per x trascurabile e possiamo scrivere (le unità sono state prese $kHz$): 
$\vert \frac{A(2 kHz)}{A(20 kHz} \vert = \sqrt{\frac{{f_0}^2  + 400}\frac{{f_0}^2 + 4}}$. Con n condensatore $C = 80 nF$ si ottiene una rapporto segnale rumore $\vert \frac{A(2 kHz)}{A(20 kHz} \vert \sim 7$ e una attenuazione del segnale $\vert A(2 kHz) \vert \sim \frac{1}{\sqrt{2}}$.

Abbiamo a disposizione un condensatore da $C = \pm nf$, e questo implica avere una resistenza $R = \pm k\Omega$.
La frequenza di taglio quindi è $f_0 = \pm kHz$ e da questi valori stimiamo i valori di $\vert A(2 kHz) \vert = \pm $ e di $\vert \frac{A(2 kHz)}{A(20 kHz} \vert$.

\paragraph{Misura frequenza di taglio}

Si è misurata la frequenza di taglio dall'intersezione delle due rette del fit. Chiamate le rette $y = a_1 x+b_1$ e $y = a_2 x+b_2$, il loro punto di intersezione è $f_0 = (b_1 - b_2)/(a_1 - a_2)$. 

\paragraph{Impedenza del circuito a bassa frequenza}




 
 
\paragraph{2.c Partitore con resitenze pi\`u grandi}
Montando di nuovo il partitore con le resistenze $R_1 = 3.80\pm 0.04 M\Omega$ e $R_2 = 3.95\pm 0.04 M\Omega$ si osservano i nuovi dati in Tabella~\ref{t:par2} e Figura~\ref{f:par2}
\begin{table}[h]
\centering
\begin{tabular}{|c|c|c|c|c|c|}
\hline 
VIN& $\sigma$ VIN  &VOUT	 & $\sigma$ VOUT& VOUT/VIN & $\sigma$ VOUT/VIN \\
\hline 
1.01&0.01&0.43&0.004&0.426&0.006\\
2.02&0.02&0.87&0.01&0.431&0.006\\
2.99&0.03&1.26&0.01&0.421&0.006\\
3.95&0.04&1.68665&0.02&0.427&0.006\\
5.01&0.05&2.13927&0.02&0.427&0.006\\
7.5&0.08&3.2025&0.03&0.427&0.006\\
10.02&0.10&4.27854&0.04&0.427&0.006\\
\hline 
\end{tabular} 
\caption{Partitore di tensione. Tutte le tensioni in V.\label{t:par2}}
\end{table}
\begin{figure}
\centering
\includegraphics[scale=0.4]{part2.pdf}
\caption{Partitore di tensione con resistenze da circa 1M.\label{f:par2}}
\end{figure}

Si osserva come valore del rapporto misurato con le resistenze da $4~M\Omega$ si discosti da quanto atteso   $V_\mathrm{OUT}/V_\mathrm{IN} = \frac{1}{1+R_1/R_2}= 0.510\pm 0.003 $. La ragione della discrepanza \`e da ricercarsi nella impedenza di ingresso del tester.


\paragraph{2.d Resistenza di ingresso del tester}

 Usando il modello mostrato nella scheda si ottiene
\[ \frac{R_1}{R_T} =  \frac{V_{IN}}{V_{OUT}} - (1 +  \frac{R_1}{R_2} )
\]
L'errore sul secondo membro \`e: 1.4\% sul primo termine, 0.7\% sul secondo termine. Entrambi i termini sono circa 2, per cui l'errore totale \`e $ 0.03 \oplus 0.015 = 0.035$, dominato dalla misura di tensione. Quindi se  $R_T > R_1/0.035 $ non abbiamo nessuna sensibilit\`a sperimentale. 
Nel primo caso risulta un numero compatibile con 0: usando VIN = 5V abbiamo $R_1/R_T =0.027 \pm 0.035 $.
Nel secondo caso risulta invece $R_1/R_T =0.38 \pm 0.035 $ cio\`e $ R_T = 10 \pm 0.9 M\Omega$.

\subsection{Partitore di corrente: 2.e}

Si monta il circuito indicato con i valori di resistenza misurati con il multimetro digitale: 
$R_3= 105\pm 2 k\Omega$, $R_1 = 550\pm 5 \Omega$, $R_2 = 230\pm 3 \Omega$.
Si fissa la tensione dell'alimentatore a $V_{IN}=10.2 \pm 0.1 V$ e si utilizza il tester digitale per misurare alternativamente la corrente nel ramo 1 e nel ramo 2, sostitendo il ramo non sotto misura con un cortocircuito. 

\rem{NOTA BENE: nelle misure di corrente \`e importante prima fare le connessioni e poi accendere l'alimentatore, per cui bisogna sempre spegnere l'alimentatore prima di modificare le connessioni.}

Si ottengono le seguenti misure: $I1 = xx \pm y \mu ~A $, $I2 = xx \pm y \mu ~A $. 
Si ripetono le misure utilizzando il tester analogico, e si ottengono i seguenti valori:
$I1 = xx \pm y \mu A $, $I2 = xx \pm y \mu ~A $. 
Ci si aspetterebbe che il rapporto tra le correnti sia $I1/I2 = R2/R1 = 0.418 \pm 0.006$ e che la somma delle correnti sia $I1 + I2 = I_{TOT} \equiv V_{IN}/R3 = 97 \pm 2 \mu~A$, considerando che l'approssimazione $I_{TOT}=V_{IN}/R3$ vale quando $R3>>$altre resistenze in gioco, ed \`e certamente verificata in questo circuito. 
Tuttavia si nota che i valori effettivamente misurati con il tester analogico ed il tester digitale si discostano da tali valori:

\begin{center}
\begin{tabular}{|c|c|c|c|c|c|c|c|c|}
\hline 
strumento& I1 ($\mu A$)& $\sigma$(I1) ($\mu A$) & I2	($\mu A$) & $\sigma$(I2) ($\mu A$) & I1/I2 
& $\sigma$(I1/I2) & I1+I2 & $\sigma$(I1+I2) \\
\hline 
Analogico & xx & xx  & xx & xx & xx & xx & xx & xx \\
Digitale & xx & xx  & xx & xx & xx & xx & xx & xx \\
\hline 
\end{tabular} 
\end{center}

La discrepanza nasce dalla resistenza interna dell'amperometro che altera la 
resistenza lungo ciascun ramo quando viene inserito. Detta $R_A$ la resistenza dell'amperometro,
questa viene sommata alternativamente ad $R1$ oppure $R2$, per cui $I1/I2 = (R2+R_A)/(R1+R_A)$ 
e $I1+I2 = I_{TOT}\cdot(R1+R2)/(R1+R2+R_A)$.
Si pu\`o quindi stimare 
$$
R_A = (R1+R2)\left(\frac{I_{TOT}}{I1+I2} - 1 \right)
$$


\section{Uso dell'oscilloscopio}

\paragraph{Misure di tensione}

\paragraph{Impedenza di ingresso dell'oscilloscopio}

\section{Misure di frequenza e tempo}

\section{Trigger dell'oscilloscopio}

\section{Conclusioni e commenti finali}
Di questa esperienza non abbiamo capito molto, per\`o \`e stato divertente far saltare i fusibili. 


\end{document}