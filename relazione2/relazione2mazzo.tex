\documentclass[10pt,a4paper]{article}
\usepackage[utf8]{inputenc}
\usepackage[italian]{babel}
\usepackage{amsmath}
\usepackage{amsfonts}
\usepackage{amssymb}
\usepackage{graphicx}
\usepackage[left=2cm,right=2cm,top=2cm,bottom=2cm]{geometry}
\newcommand{\rem}[1]{[\emph{#1}]}

\author{Gruppo AC \\ Federico Belliardo, Francesco Mazzoncini, Giulia Franchi}
\title{Esercitazione N.2: Circuito RC - Filtri Passivi.}
\begin{document}

\maketitle

\section{Scopo e strumentazione}

Misurare la frequenza di un filtro passa-basso e studiare la variazione della risposta del filtro in funzione del carico applicato a valle. In seguito studiare la l'attenuazione di un filto passa-banda.
\rem{Aggiungere strumentazione}

\section{Filtro passa-basso}

\paragraph{Progettazione filtro.}
Si vogliono trovare i valori dei componenti resistivo e capacitivo del filtro perchè trasmetta un segnale sinusoidale di frequenza $2kHz$ e attenui il rumore a $20kHz$.

\begin{figure}[h]
\centering
\includegraphics{passabasso.png}
\caption{Schema del circuito passa-basso}
\end{figure}

Risolvendo il circuito e chiamando $r$ la resitenza di carico e $R$ la resistenza del passabasso si ottiene la seguente relazione per il modulo dell'attenuazione:

$\vert A(\omega) \vert = \frac{1}{\sqrt{(1+x)^2+(\frac{f}{f_{0}})^2}} $. Dove $f_0$ è la frequenza di taglio del filtro. Definite $f_2 = 20 kHz$ e $f_1 = 2 kHz$ le frequenze del rumore e del segnale e $x = \frac {R}{r}$ otteniamo come rapporto di attenuazione:
$\vert \frac{A(2 kHz)}{A(20 kHz} \vert = \sqrt{\frac{{f_0}^2 (1+x)^2 + {f_2}^2}{{f_0}^2 (1+x)^2 + {f_1}^2}}$

Selezionando una resistenza $R = 1 k\Omega$ del filtro molto minore del carico $r = 100 k\Omega$ otteniamo $x = 0.01$ cioè un valore per $x$ trascurabile e possiamo scrivere (le unità sono state prese $kHz$):
$\vert \frac{A(2 kHz)}{A(20 kHz} \vert = \sqrt{\frac{{f_0}^2  + 400}{{f_0}^2 + 4}}$. Si sceglierà la frequenza di taglio del filtro essere $f_0 = 2kHz$. Con un condensatore $C = 80 nF$ si ottiene una rapporto segnale rumore $\vert \frac{A(2 kHz)}{A(20 kHz} \vert \sim 7$ e una attenuazione del segnale $\vert A(2 kHz) \vert \sim \frac{1}{\sqrt{2}}$.

Abbiamo a disposizione un condensatore da $C =70 \pm3 nF$, e questo implica avere una resistenza $R =1,20 \pm 0,01 k\Omega$.
La frequenza di taglio quindi è $f_0 =1,90 \pm 0,08kHz$ e da questi valori stimiamo i valori di $\vert A(2 kHz) \vert = 0,68\pm0,02 $ e di $\vert \frac{A(2 kHz)}{A(20 kHz} \vert=7,3\pm0,1$.

\paragraph{Misura frequenza di taglio.}
Come prima stima della frequenza di taglio abbiamo esplorato le frequenze in cui $\ V_{out}=\frac{V_{in}\pm\sigma V_{in}}{\sqrt{2}}$ con incertezza la loro semidispersione, ottenendo *****.
Abbiamo poi preso varie misure del segnale in uscita con frequenze comprese tra $100 Hz $ e $1 MHz$.
Riportiamo i dati relativi nella \ref{tab1}

Abbiamo eseguito 3 fit numerici:
\begin{enumerate}
	\item Abbiamo eseguito un fit numerico con una costante  per il range di frequenze in cui il filtro passabasso non attenua il segnale in entrata. La retta  come da aspettativa è concorde con $0 dB$ $(A=0.22\pm0.1 dB)$  entro $3\sigma$.
 come è evidente dal grafico del fit il $\chi^2 $risulta essere molto piccolo:$\frac{\chi^2}{ndof}=0.65$

\begin{figure}[!htb]
  \centering
  \includegraphics[scale=0.7]{orizz}
\caption{Fit orizzontale}
\end{figure}
  

\item  fit con una retta a due parametri $y=mx+q $per il range di frequenze  in cui il filtro passabasso attenua il segnale in entrata,ottenendo $m=-18.1\pm0.3$ con matrice di covarianza$ \left[\begin{matrix}0,12 &-0,46 \\ -0,46 & 1,73\end{matrix}\right]$



\begin{figure}[!htb]
  \centering
  \includegraphics[scale=0.5]{discesa}
\caption{Fit attenuazione}
\end{figure}
Si è misurata la frequenza di taglio dall'intersezione delle due rette del fit. Chiamate le rette $y = a_1 x+b_1$ e $y = a_2 x+b_2$, il loro punto di intersezione è $f_0 = \frac{b_1 - b_2}{a_1 - a_2}$.
Considerando che $a_1=0$ e quindi  $f_0 = \frac{b_1 - b_2}{- a_2}$, otteniamo $f_0 =$


\item Abbiamo eseguito un fit numerico su tutti i dati con la funzione di trasferimento  $\vert A(f) \vert = \frac{1}{\sqrt{1+(\frac{f}{f_{0}})^2}} $, ottenendo $f_t=1960 \pm 60 Hz$ ($\frac{\chi^2}{ndof}=0.15$)
\begin{figure}[!htb]
  \centering
  \includegraphics[scale=0.4]{fitcompleto}
\caption{Fit funzione di trasferimento}
\end{figure}
\end{enumerate}


\paragraph{Impedenza del circuito a bassa frequenza.}
L'impedenza di ingresso di un circuito come quello disegnatio all'inizio della relazione è $Z_{ingresso} = R+\frac{r}{j \omega C r+1}$ , dunque a bassa frequenza in condensatore è un aperto dunque l'impedenza di ingresso è $R+r$, mentre ad alta frequenza è un cortocircuito dunque l'impendenza è $R$. Alla frequenza di taglio abbiamo poi: $Z_{ingresso} = R + \frac{Rr}{jr+R}$. L'efetto della resistenza di carico sul circuito è di diminuirne il guadagno, secondo la formula: $\vert A(\omega) \vert = \frac{1}{\sqrt{(1+x)^2+(\frac{f}{f_{0}})^2}} $.
Lo scostamento della risposta da quella del filtro ideale è tanto maggiore quanto il carico resistivo è più vicino alla resistenza del filtro.



\section{Filtro passa-banda}

\begin{figure}[h]
\centering
\includegraphics[width=0.5\textwidth]{passabanda.png}
\caption{Filtro passa banda}
\end{figure}

\paragraph{Verifiche sui circuiti passa alto e passa basso.}

\paragraph{Misure sul fitro passa banda.}
%misure delle frequenze di taglio e dell'ampiezza massima (dovrebbe essere -6 dB)

\paragraph{Spegazioni teoriche.}
La funzione di trasferimento teorica per un circuito passa banda come quelli disegnati sopra è:
$V_{out} = A_{1} A_{2} \frac{Z_{in}^2}{Z_{out}^2+Z_{in}^2} V_{in}$, dove gli apici si riferiscono a al primo o al secondo circuito in sequenza.

La seguente tabella riassume le impedenze di ingresso e sucita per circuiti passa basso:

\begin{table}[h]
\centering
\begin{tabular}{|c|c|c|}
\hline
 & Passa-basso  & Passa-alto \\
\hline
Ingresso & $R+\frac{1}{j \omega C}$ & $R+\frac{1}{j \omega C}$\\
Uscita & $AR$ & $j \omega C A$\\
\hline
\end{tabular}
\caption{Riassunto resistenze di ingresso e di uscita.}
\end{table}

Semplificando l'espressione otteniamo: $V_{out} = A_{1} A_{2} \frac{1}{1+\frac{R_1}{R_2} A_1 A_2} V_{in}$.
E nel nostro caso (due resistenze uguali otteniamo:
$V_{out} = A_{1} A_{2} \frac{1}{1+A_1 A_2} V_{in}$, che è limitato dall'alto da $A_{max} = \frac{1}{2}$.
Se chiamiamo $\omega_1$ la frequenza di taglio del circuito passa alto e $\omega_2$ la frequenza di taglio del circuito passa alto e la frequenza del cicruito passa basso otteniamo i seguenti limiti per l'attenuazione:
\begin{itemize}
\item $\omega \ll \omega_1$ allora $A_1 \sim 1$ dunque
$A_{tot} = \frac{A_2}{1+A_2}$, sviluppando otteniamo
$A_{tot} = \frac{1}{2} \frac{1}{1-j \frac{\frac{\omega_2}{2}}{\omega}}$ dunque il filtro è equivalente a un passa alto con frequenza di tagio $\frac{\omega_2}{2}$.
\item $\omega \gg \omega_2$ allora $A_2 \sim 1$ dunque
$A_{tot} = \frac{A_1}{1+A_1}$, sviluppando otteniamo:
$A_{tot} = \frac{1}{2} \frac{1}{1+j \frac{omega}{2 \omega_1}}$.
\end{itemize}

Se vogliamo che $A_{tot} = A_1 A_2$ deve essere $R_1 \ll R_2$ come è evidente.

\end{document} 